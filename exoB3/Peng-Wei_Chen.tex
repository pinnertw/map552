\documentclass[12pt]{article}
\usepackage{fullpage,enumitem,amsmath,amssymb,graphicx}

% Pour avoir les accents
\usepackage[utf8]{inputenc}  
\usepackage[T1]{fontenc}   

% Some macros for your convenience
\newcommand\bbR{\ensuremath{\mathbb{R}}} % Real numbers
\newcommand\bbZ{\ensuremath{\mathbb{Z}}} % Integers
\newcommand\bbE{\ensuremath{\mathbb{E}}} % Expectation
\DeclareMathOperator*{\tr}{tr} % Trace
\DeclareMathOperator*{\diag}{diag} % Diagonal matrix
\DeclareMathOperator*{\sign}{sign} % Sign
\DeclareMathOperator*{\var}{Var} % Variance
\DeclareMathOperator*{\cov}{Cov} % Covariance
\DeclareMathOperator*{\argmin}{Argmin} % \argmin

\newcommand{\1}{\mathbb{I}} % Indicator
%\newcommand{\newproblem}[1]{\newpage \section*{Problem #1}}

\begin{document}

\begin{center}
{\Large MAP 552 exercice B3}

\begin{tabular}{c}
Peng-Wei Chen\\
\end{tabular}
\end{center}

\section*{Question 1}
We can use the theorem 6.12. Since $\varphi$ in $\mathbb{H}^2$. We only need to show that, for all $t \le 0$, we have $\int_0^t \frac{\varphi_s}{||\varphi_s||_n} \frac{\varphi_s^T}{||\varphi_s||_n} ds= tI_n = t$ (n = 1).
\[
    = \int_0^t \frac{\varphi_s \varphi_s^T}{|\varphi_s^1|^2 + \cdot + |\varphi_s^n|^2} ds = \int_0^t 1ds = t
\]
Therefore, $W_t$ is a Brownian motion by theorem 6.12..

\section*{Question 2}
We apply the Ito's formula on 
\[
    X_t = \sum_{i=1}^{n} \left(x_i + B_t^i\right)^2, \left( x_1,\dots,x_n \right) \in \mathbb{R}^n
\]
With $f(t, x) = ||z + x||^2_2, z = \left( x_1,\dots,x_n \right)$, we have the equation
\[
    f(t, B_t) = f(0, 0) + \int_{0}^{t} Df(u, B_u)\cdot dB_u + \int_{0}^{t}\left(f_t + \frac{1}{2} Tr\left[ D^2f \right]\right)(u, B_u)du
\]
We have
\[
    \begin{array}[ht!]{r l}
        \partial_t f(t, x) &= 0\\
        Df(t, x) & = 2(z + x)\\
        D^2f(t, x) & = 2I_n\mbox{, }I_n\mbox{ is the diagonal matrix}\\
    \end{array}
\]
, so
\[
    f(t, B_t) = f(0, 0) + \int_{0}^{t} 2(z + B_s)\cdot dB_s + \int_{0}^{t} (0 + n)dt = ||z||_2^2 + \int_{0}^{t}ndt + \int_0^t 2B_s\cdot dB_s
\]
We identify
\[
    \begin{array}[ht!]{r l}
        X_0 & = ||z||_2^2\\
        \phi_s & = 2(z + B_s)\\
        \psi_s & = n\\

    \end{array}
\]
For $\mathbb{E}\left[ \int_0^T||\phi_s||_n^2ds \right] < +\infty$, by Fubini's theorem :
\[
    = \int_0^T \mathbb{E}\left[||2(z + B_s)||_n^2\right] ds = 4\int_{0}^{T} z\cdot z + ns ds = 4Tz\cdot z +2nT^2 < \infty \mbox{ for T > 0}
\]
because $B_s \sim N\left( 0, sI_n \right)$ in distribution.

\section*{Question 3}
We define 
\[
    d\beta_t = \frac{1}{\sqrt{X_t}}(B_t + z)\cdot dB_t, \beta_0 = 0
\]
Therefore, we have
\[
    \beta_t = \int_0^t \frac{(B_t + z)\cdot dB_t}{\sqrt{X_t}}
\]
Define $\varphi_s = z + B_t$, then $||\varphi_s||_n = \sqrt{X_t}$. By question 1, $(\beta_t)$ is thus a Brownian motion. 
\end{document}

